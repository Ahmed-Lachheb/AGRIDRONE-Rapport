\documentclass[a4paper,12pt]{report}
\usepackage{graphicx}
\usepackage[T1]{fontenc}
\usepackage[utf8]{inputenc}


\usepackage[french]{babel}



\begin{document}

\subsection {Action proportionnelle}
L’effet de l’action proportionnelle consiste à amplifier l’erreur d’un gain constant afin que le
système réagisse plus rapidement aux changements de consigne. Cette action proportionnelle est
représentée comme suit :
$ 𝐶(𝑡) = 𝐾_𝑝  \times \epsilon(t)$
Plus la valeur de 𝐾𝑝 est grande plus la réponse est rapide mais au détriment d’une détérioration
de la stabilité du système allant jusqu’à l’instabilité pour de grandes valeurs.




\subsection  {Action intégrale}
L’action intégrale a pour but de réduire voire d’éliminer l’erreur statique en régime permanent
pour réaliser cela le régulateur intègre l’erreur par rapport au temps et multiplie le résultat par une
constante 𝐾𝑖 comme suit :
$ 𝐶(𝑡) = K_i\int\epsilon(t)dt$

Plus la valeur de 𝐾𝑖 est grande plus l’erreur statique sera vite corrigée mais nous perdons un peu
en stabilité et il y a un risque de dépassement qui subvient.

\subsection  {Action dérivée}
Pour obtenir une action dérivée nous multipliant la dérivée de l’erreur par un coefficient 𝐾𝑑
cette action permet d’éliminer le dépassement de la réponse et d’améliorer la stabilité du système. Sa
relation est donnée comme suit :
$ 𝐶(𝑡) = K_d \frac{d\epsilon(t)}{dt}$
Plus la valeur de 𝐾𝑑 est grand plus le dépassement est atténué mais si elle est trop grande le
système est ralenti jusqu’à risquer de devenir instable pour de très grandes valeurs.




\subsection {Action PID} 
Le régulateur PID combine les trois actions vues précédemment et permet ainsi d’avoir de
bonnes performances aussi bien en vitesse en stabilité qu’en précision. L’expression d’un régulateur
PID est donné comme suit :
$ 𝐶(𝑡) = 𝐾_𝑝  \times \epsilon(t) + K_i\int\epsilon(t)dt + K_d \frac{d\epsilon(t)}{dt} $
Ainsi ce régulateur permet d’atteindre les objectifs fixés en termes de vitesse de stabilité et de
précision et ce en trouvant la configuration optimale des valeurs des différents gains 𝐾𝑝, 𝐾𝑖 et 𝐾𝑑.	





\subsection {Action proportionnelle}
L’effet de l’action proportionnelle consiste à amplifier l’erreur d’un gain constant afin que le
système réagisse plus rapidement aux changements de consigne. Cette action proportionnelle est
représentée comme suit :

$ c(𝑡) = 𝐾_𝑝  \times \epsilon(t) $


Plus la valeur de $𝐾_𝑝 $ est grande plus la réponse est rapide mais au détriment d’une détérioration
de la stabilité du système allant jusqu’à l’instabilité pour de grandes valeurs.




\subsection  {Action intégrale}
L’action intégrale a pour but de réduire voire d’éliminer l’erreur statique en régime permanent
pour réaliser cela le régulateur intègre l’erreur par rapport au temps et multiplie le résultat par une
constante $K_i$ comme suit :

$ c(𝑡) = K_i \int\ epsilon(t) dt $


Plus la valeur de $K_i$ est grande plus l’erreur statique sera vite corrigée mais nous perdons un peu
en stabilité et il y a un risque de dépassement qui subvient.

\subsection  {Action dérivée}
Pour obtenir une action dérivée nous multipliant la dérivée de l’erreur par un coefficient $ K_d $
cette action permet d’éliminer le dépassement de la réponse et d’améliorer la stabilité du système. Sa
relation est donnée comme suit :

$ c(𝑡) = K_d \frac{d\epsilon(t)}{dt} $


Plus la valeur de 𝐾𝑑 est grand plus le dépassement est atténué mais si elle est trop grande le
système est ralenti jusqu’à risquer de devenir instable pour de très grandes valeurs
	
	
\end{document}