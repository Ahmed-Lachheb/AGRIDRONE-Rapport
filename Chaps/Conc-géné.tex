\chapter*{\centering Conclusion Générale }
\markboth{CONCLUSION}{}
Notre projet de fin d'études a été une opportunité pour développer nos compétences en mécatronique et découvrir le domaine de l'intelligence artificielle.


Nous avons étudié tout d'abord notre idée qui consiste à réaliser un drone et nous avons donc constaté que l'agriculteur souffre d'un manque au niveau de l'implémentation des nouvelles technologies particulièrement dans la détection et le comptage de ses fruits. Pour cela, nous avons décidé d'orienter notre projet vers le secteur de l'agriculture et de travailler spécialement sur la détection des oranges. 


Dans le but de modéliser notre système, nous avons travaillé avec les diagrammes SysML tels que les diagrammes de cas d'utilisation et d'exigence pour décrire les besoins fonctionnels ainsi que les diagrammes de définition de blocs pour exprimer la structure de notre système. Avant de passer à la réalisation de notre AGRIDrone, nous avons effectué une recherche selon les types de drones ainsi qu'une étude comparative entre eux afin de choisir celui qui est convenable à nos besoins. 
En plus, nous avons effectué une étude théorique sur l'intellignece artificielle pour étudier la méthode de détection des oranges. 
Finalement, au cours de la réalisation de notre drone et lors de la configuration et la calibration de l'APM2.8 et la radiocommande au niveau de laquelle nous avons rencontré un grand problème, ce qui nous a empêché de mieux avancer dans notre travail puisqu'il a pris beaucoup de temps pour le résoudre.

En guise de perspectives, nous prévoyons compléter la partie manquante dans notre réalisation et améliorer les performances de notre système à travers la généralisation du processus de détection. En plus, nous souhaitons intervenir de nouveaux outils pour que notre nouveau système sera capable
de cueillir les fruits.
