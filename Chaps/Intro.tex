\chapter*{Introduction générale}
\addcontentsline{toc}{chapter}{Introduction générale}
%\markboth{Introduction générale}{}
La fusion de la composante logicielle (software) dans la composante matérielle (hardware) donne naissance généralement aux systèmes embarqués.
Selon les exigences, ces systèmes peuvent subir des adaptations matérielle ou logicielle. Il s'agit d'un système autonome qui doit réaliser son travail en tenant compte de plusieurs contraintes : sa taille, le temps d'exécution dont il dispose et l'énergie \cite{FUTURA}. Les systèmes embarqués sont très utilisés en aéronautique et dans l'industrie automobile. On les trouve également dans le domaine militaire à l'intérieur des missiles et bien évidemment dans l'agriculture, dans les distributeurs automatiques de billets, dans les équipements médicaux, dans les imprimantes \ldots
Depuis 2007, le secteur des systèmes embarqués a connu une croissance de $9,9\%$ de ses effectifs par an  et une croissance qui devrait encore évoluer dans les prochaines années\cite{PierreAudoin2012}. 
Un système embarqué doit être capable d'exécuter des tâches en temps réel. 
Pour ce faire, il est équipé de capteurs, d'actionneurs et d'une interface homme machine (IHM). 
L'innovation des drones fait partie des systèmes embarqués les plus marquantes. 

Développé par l'ingénieur et l'auteur anglais Archibald Low en 1916 pour des besoins de l'armée durant la première guerre mondiale, le drone est en tête de liste des systèmes embarqués ayant une très forte demande \cite{STUDIOFLY}. 
De nos jours, cette innovation, nommée aussi les véhicules aériens sans pilote, est également un secteur en forte croissance. 
Ses avantages sont indéniables grace à la réduction des coûts par rapport aux hélicoptères, à l'élimination des risques de mise en danger de vies humaines, et à la rapidité de sa mise en œuvre \cite{AltiGator}. 
Les drones peuvent emporter une caméra, capable de retransmettre en temps réel ce qui se passe sur le terrain. 
Ou emporter une caméra infrarouge, détectant les sources de chaleur humaine, animale, un feu, un moteur, \ldots .
\corr{Parler des réseaux de drones avec références.}
Le calculateur transporté par le drone gangne de puissance au fil des années. Ces calculateurs ouvrent les portes grandes larges aux applications des drones en les octroyant la puissance minimale requise permettant l'exécution des programmes exigeants en termes de ressources (puissance de calcul et mémoire), notamment ceux de l'intelligence artificielle (IA). En effet, les drones sont maintenant intélligents : ils peuvent invoquer un réseau de neurones.

\corr{Changer ce paragraphe pour parler des RNA}En outre, l'IA consiste à mettre en œuvre un certain nombre de techniques visant à permettre aux machines d'imiter une forme d'intelligence réelle. L'IA se retrouve implémentée dans plusieurs domaines d'application\cite{netactions} y compris surtout le domaine de l'agriculture. Elle assure le passage de l'agriculture traditionnelle vers l'agriculture de précision où la gestion de la production optimale. En effet, selon le dictionnaire \cite{leshorizons}, le principe de l'agriculture de précision consiste à utiliser les nouvelles technologies tels que l'intelligence artificielle, la robotique\ldots\ Cela pour augmenter les rendements d'une parcelle et d'optimiser le travail des agriculteurs.

Notre projet de fin d'études s'installe dans un cadre d'agriculture de précision dans lequel nous désirons aider l'agriculteur à estimer sa production avec un drone intélligent capable de reconnaitre les fruits. L'utilisation de drones dans l'agriculture de précision permet de gérer la variabilité spatiale et temporelle pour améliorer les rendements économiques. Afin de s'organiser de manière appropriée, les agriculteurs ont besoin d'anticiper leur production notament que le manque d'une telle préditon de la récolte se traduit par des problèmes de distribution et de commercialisation.

C'est pourquoi le présent rapport est constitué de trois chapitres. 
Dans le premier, nous présenterons le cadre de projet y compris la problématique, l'objectif, la spécification des besoins et le cahier de charge. 
\corr{\`A reformuler} Le deuxième chapitre portera sur l'état de l'art de notre système qui sera expliqué en deux grandes parties. La première concerne le drone et la deuxième porte sur l'intelligence artificielle. Dans le dernier chapitre, nous illustrerons la phase de la réalisation de notre solution.