
\chapter{Introduction générale}
La fusion de la composante logicielle (software) dans la composante matérielle (hardware) donne naissance généralement aux systèmes embarqués.
Selon les exigences, ces systèmes peuvent subir des adaptations matérielle ou logicielle. Il s'agit d'un système autonome qui doit réaliser son travail en tenant compte de plusieurs contraintes : sa taille, le temps d'exécution dont il dispose et l'énergie [1]. Les systèmes embarqués sont très utilisés en aéronautique et dans l'industrie automobile. On les trouve également dans le domaine militaire à l'intérieur des missiles et bien évidemment dans l'agriculture, dans les distributeurs automatiques de billets, dans certains matériels médicaux, dans les imprimantes\ldots\  Depuis 2007, le secteur des systèmes embarqués a connu une croissance de $9,9\%$ de ses effectifs par an[2] et une croissance qui devrait encore évoluer dans les prochaines années. Un système embarqué doit être capable d'exécuter des tâches en temps réel. Pour ce faire, il est équipé de capteurs, d'actionneurs et d'une interface. L'innovation des drones fait partie des systèmes embarqués les plus fréquents. 

%\corr{Ce paragraphe est à déplacer pour le moment}
%ces systèmes appartiennent à la grande famille de l'intelligence artificielle faible. Cette intelligence est dite « faible » car le système n'est pas capable de ressentir des émotions ni de communiquer avec les êtres humains, contrairement à l'intelligence artificielle forte. Le système embarqué se contente de réaliser des tâches prédéfinies [1] et sa création permet aux drones par exemple de communiquer correctement au sein de la flotte de drones. 

Commençant par un aperçu sur l'historique du drone qui est développé par l'ingénieur et l'auteur anglais Archibald Low en 1916 pour les besoins de l'armée durant la première guerre mondiale[22]. 

De nos jours, cette innovation qui est nommée par les véhicules aériens sans pilote (UAV ou drones) est également un secteur en forte croissance, grâce aux avantages indéniables de la réduction des coûts par rapport aux hélicoptères, de l'élimination des risques de mise en danger de vies humaines dans certains cas, et de la rapidité de mise en œuvre[3]. Les drones peuvent emporter une caméra, capable de retransmettre en temps réel ce qui se passe sur le terrain, une caméra infrarouge, détectant la chaleur (humaine, animale, d'un feu, d'un moteur, etc.), un gyroscope permettant de stabiliser les mouvements du drone, améliorant le suivi d'une cible ou la qualité d'une image ainsi que la possibilité de former un ensemble de drones à intelligence collective, pour des applications civiles ou militaires. 

En outre, l'intelligence artificielle (l'IA) consiste à mettre en œuvre un certain nombre de techniques visant à permettre aux machines d'imiter une forme d'intelligence réelle. L'IA se retrouve implémentée dans plusieurs domaines d'application[4] y compris surtout le domaine de l'agriculture. Elle assure le passage de l'agriculture traditionnelle vers l'agriculture de précision où la gestion de la production optimale. En effet, selon le dicionnaire [21], le principe de l'agriculture de précision consiste à utiliser les nouvelles technologies tels que l'intelligence artificielle, la robotique,\ldots\ Cela pour augmenter les rendements d'une parcelle, d'optimiser le travail des agriculteurs et réduire la consommation d'énergie, d'eau et d'intrants.

Notre projet de fin d'études s'installe dans un cadre d'agriculture de précision dans laquelle on veut aider l'agriculteur à estimer sa production en combinant le système embarqué et l'intelligence artificielle. Le drone en tant qu'un système embarqué peut servir à résoudre certains problèmes urgents auxquels sont confrontés l'agriculture et l'environnement. L'utilisation de drones dans l'agriculture de précision permet de gérer la variabilité spatiale et temporelle pour améliorer les rendements économiques. Afin de s'organiser de manière appropriée, les agriculteurs ont besoin d'anticiper leur production nottament que le manque d'une telle préditon de la récolte se traduit par des problèmes de distribution et de commercialisation.

Notre rapport est constitué de trois chapitres. Dans le premier, nous présenterons le cadre de projet y compris la problématique, l'objectif, la spécification des besoins et le cahier de charge. Le deuxième chapitre porte sur l'état de l'art de notre système et dans le dernier chapitre nous illustrerons la phase de la réalisation de notre projet.


	
	
	
	
	
	
	
