
\chapter{Introduction générale}
	
Le système embarqué est à la fois un matériel et un logiciel utilisé en informatique et en électronique, avec des applications dans de nombreux domaines. Il s'agit d'un système autonome qui doit réaliser son travail en tenant compte de plusieurs contraintes : sa taille, le temps d'exécution dont il dispose et l'énergie [1]. Les systèmes embarqués électriques sont très utilisés en aéronautique et dans l'industrie automobile. On les trouve également dans le domaine militaire à l'intérieur des missiles et bien évidemment dans l’agriculture, dans les distributeurs automatiques de billets, dans les consoles de jeux, dans certains matériels médicaux, dans les appareils électroménagers, dans les imprimantes... Et bien entendu, dans les ordinateurs, les téléphones portables et les disques durs… Depuis 2007, le secteur des systèmes embarqués a connu une croissance de 9,9\%  de ses effectifs par an[2], une croissance qui devrait encore évoluer dans les prochaines années. Un système embarqué doit être capable d'exécuter des tâches en temps réel. Pour ce faire, il est équipé de capteurs, d'actionneurs et d'une interface. Les systèmes embarqués appartiennent à la grande famille de l'intelligence artificielle faible. Cette intelligence est dite « faible » car le système n'est pas capable de ressentir des émotions ni de communiquer avec les êtres humains, contrairement à l'intelligence artificielle forte. Le système embarqué se contente de réaliser des tâches prédéfinies [1]. La création de ce système, permet aux drones par exemple de communiquer correctement au sein de la flotte de drones. 


Cette innovation qui est nommé par les véhicules aériens sans pilote (UAV ou drones) est également un secteur en forte croissance, grâce aux avantages indéniables de la réduction des coûts par rapport aux hélicoptères, de l’élimination des risques de mise en danger de vies humaines dans certains cas, et de la rapidité de mise en œuvre[3]. Les drones peuvent emporter ; une caméra, capable de retransmettre en temps réel ce qui se passe sur le terrain, une caméra infrarouge, détectant la chaleur (humaine, animale, d'un feu, d'un moteur, etc.), un gyroscope permettant de stabiliser les mouvements du drone, améliorant le suivi d'une cible ou la qualité d'une image ainsi que la possibilité de former un essaim de drones à intelligence collective, pour des applications civiles ou militaires. Les drones s’observent de quatre manières différentes : visuelle, radar, infrarouge, radioélectrique et électro-optique, et peuvent garder une trace photographique ou/et vidéo.


En outre, l’intelligence artificielle (l’IA) consiste à mettre en œuvre un certain nombre de techniques visant à permettre aux machines d'imiter une forme d'intelligence réelle. L'IA se retrouve implémentée dans un nombre grandissant de domaines d'application[4] y compris le domaine de l’agriculture de précision spécifiquement parce qu’avec la demande croissante de nourriture au sein de nos sociétés, l’augmentation de la population, le problème de l’insuffisance hydrique et l’impossibilité d’utiliser de nouveaux champs pour les cultures, il est essentiel d’augmenter la productivité des plantations en utilisant cette intelligence artificiel pour bien facilité et développer ce domaine-là.


Notre projet de recherche s’installe dans un cadre d’agriculture de précision dans laquelle on veut aider l’agriculteur à estimer son production en combinant un système embarqué et l’intelligence artificielle. Le drone en tant qu’un système embarqué peut servir à résoudre certains des problèmes urgents auxquels sont confrontés l'agriculture et l'environnement. L'utilisation de drones dans l'agriculture de précision permet de gérer la variabilité spatiale et temporelle pour améliorer les rendements économiques. Afin de s’organiser de manière appropriée, les agriculteurs ont besoin d’anticiper leur production. Mais le comptage des fruits est une tâche qui prend du temps.
Dans ce contexte, le drone peut détecter des alimentations et plus spécifiquement les oranges. Cette détection est déclenchée par des réseaux dits de neurones. Les réseaux de neurones artificiels permettent aux machines de copier le travail du cerveau humain. Les techniques d'intelligence artificielle se complètent progressivement jusqu'à ce que le réseau de neurones devienne autonome, grâce aux techniques d'apprentissage en profondeur.

	
	
	
	
	
	
	
	
