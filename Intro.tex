\documentclass[a4paper,12pt]{report}
\usepackage{graphicx}
\usepackage[T1]{fontenc}
\usepackage[utf8]{inputenc}
\usepackage[french]{babel}
\usepackage{blindtext}
\title{Rapport du projet de fin d'études}
\author{Ahmed Baha Eddine Lachheb et Ahmed Arbi Adouani}
\date{2022}
\begin{document}
	\title {Introduction génerale}
	
	L'utilisation de drones offre un grand potentiel pour résoudre certains des problèmes urgents auxquels sont confrontés l'agriculture et l'environnement, en particulier pour l'acquisition de données de haute qualité pouvant être utilisées en temps réel. Dans les zones arides, le secteur agricole sera l'un des plus grands utilisateurs de drones au monde dans les années à venir, avec la surveillance des écosystèmes et la désertification. Les drones permettent notamment d'améliorer la production agricole (agriculture de précision), de surveiller, surveiller, évaluer les écosystèmes et la désertification, tout en contribuant à développer des systèmes d'alerte .L'utilisation de drones ou l'agriculture de précision permettent de gérer la variabilité spatiale et temporelle pour améliorer les rendements économiques. Cela inclut des systèmes d'aide à la décision pour gérer l'ensemble de l'exploitation,  dans le but d’optimiser les rendements des intrants tout en préservant les ressources grâce à l’utilisation des images aériennes par drones (rendements des cultures, caractéristiques du terrain/topographie, teneur en matière organique, niveaux d’humidité, niveaux d’azote…).
	
	
	
	
	
	
	
	
\end{document}